%\usepackage{algorithm}
%\usepackage{algorithmic}

\section{Implementation in FreeFem++}
To solve this problem we use FreeFem++. First we have to define the problems we need to solve at each iteration. 
We have 
\begin{align}\label{eq1}
& \Delta \phi = G \\
& \left. \phi\right|_{\partial\Omega} = 0\nonumber
\end{align}
\begin{align}\label{eq2}
& \Delta \lambda = 0 \\
& \left. \lambda\right|_{\partial\Omega} = g(s)-\frac{\partial \phi}{\partial n}\nonumber
\end{align}
where g is an approximate function to $\frac{\partial \phi_0}{\partial n}$. \\
We define two problems named \textit{a} and \textit{b} that solves respectively problems (\ref{eq1}) and (\ref{eq2}).\\
Moreover, in order to get the value of g, we have to define a problem called \textit{initial} that solves $\Delta \phi_0 = G(x,x_0,y,y_0)$
\lstinputlisting[language=C++]{../problems.edp}

We use a gradient descent algorithm to find the position of the source.\\
\begin{algorithm}
\caption{Gradient algorithm}
\begin{algorithmic} 
\WHILE{$||\nabla f(x_k)|| \geqslant \epsilon$}
\STATE calculation of $\nabla f(x_k)$.
\STATE calculation of $\alpha_k$
\STATE $x_{k+1} = x_k - \alpha_k \nabla f(x_k)$
\ENDWHILE
\end{algorithmic}
\end{algorithm}

We know that $\frac{\partial I}{\partial x_0} = -\int_{\Omega}\lambda \frac{\partial G}{\partial x_0} dx$ and $\frac{\partial I}{\partial y_0} = -\int_{\Omega}\lambda \frac{\partial G}{\partial y_0} dx$. So we need to calculate $\nabla G(x_0,y_0)$ and $\lambda$. To get $\lambda$, we must calculate the problem \textit{b} and to do that we need to calculate the problem \textit{a}.

Then we have to calculate the coefficient $\alpha_k$

\lstinputlisting[language=C++]{../gradient.edp}