\section{Setting}
To begin, we specify the problem. Let $\Omega = B_0(1)$ be the unit disc in $\mathbb R^2$. A heat distribution with source $G\in C^1(\Omega)\cap C^0(\overline{\Omega})$ is a function $\phi\in H^1_0(\Omega)$ satisfying
\begin{equation}\label{eq:1}
	\Delta \phi = G 
	\hspace{30pt}\text{in }\Omega.
\end{equation}
Note that other choices for boundary values are also possible in principle, however they might necessitate changes for resulting formulations in the later chapters.

We presuppose that any heat source must have the form
\begin{equation}\label{eq:2}
	G(x_0,y_0) = \exp\left((x-x_0)^2+(y-y_0)^2\right),\text{ for all }x,y\in\overline{\Omega}
\end{equation}
i.e. that heat is radiating from a single point $(x_0,y_0)\in\Omega$. We suppose further that this point is unknown to the observer who can only measure the heat flux $\frac{\partial \phi}{\partial n}$ of the resulting distribution $\phi$ at the boundary $\partial\Omega$.

His task thus is: given a (potentially approximate) heat flux $g\in C^0(\partial\Omega)$, find the optimal solution $\left(\phi^*,(x^*,y^*)\right)\in \mathcal C\times\Omega$ to the minimization problem
\begin{align}\label{eq:3}
	 \underset{\mathcal C\times\Omega}{\min}\;I(\phi,(x_0,y_0))
	&=\frac{1}{2} 
		\int_{\partial\Omega}
			\left( \frac{\partial \phi^*}{\partial n} - g\right)^2 \mathrm{d}s\\
	\text{with }\Delta \phi^* 
	&= G(x^*,y^*)\label{eq:4}
\end{align}
for the heat distribution space $\mathcal C = \{ \phi\in H^1_0(\Omega)\:|\:\Delta \phi = G\text{ in }\Omega,\:\text{ with $G$ of the form (\ref{eq:2})}\}$.